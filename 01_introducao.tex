\chapter{Introdução}

As aplicações Web acompanham a vida acadêmica dos estudantes da Universidade Federal da Bahia. Diversas aplicações trazem um conjunto de informações que são relevantes para a trajetória do estudante dentro da universidade, como por exemplo, disciplinas ofertadas por semestre, calendário acadêmico, regimentos e grade curricular. Contudo, a universidade não provê nenhum sistema que se preocupe em transparecer para o discente o quão próximo ele está da formatura, ou o que é necessário para alcançá-la. Nesse cenário, surge o MeForma.

A ideia do MeForma é tornar o estudante mais participativo e preocupado com a sua formação, permitindo que ele possa acompanhar seu desempenho no curso a partir de seus dados acadêmicos, como disciplinas cursadas e cargas horárias aproveitadas. Esses dados são inseridos pelos próprios usuários, computados pelo sistema e exibidos em forma de porcentagem de completude do curso.

O ponto de partida do MeForma foi uma planilha, criada em 2015, que servia de apoio a dois estudantes do curso de Ciência da Computação da UFBA. Através da planilha, é possível acompanhar a porcentagem de disciplinas obrigatórias concluídas, a porcentagem de carga horária optativa, e a porcentagem de carga horária complementar. Em 2016, foi lançado um aplicativo para o sistema Android com a intenção de compartilhar os benefícios trazidos pela utilização da planilha com todos os estudantes da UFBA. E, em 2018, surgiu o desafio de reconstruir o MeForma com base nas necessidades e melhorias apontadas pelos usuários durante o tempo de vida do aplicativo já lançado. O novo sistema é chamado neste trabalho de MeForma2 e está disponível para as plataformas Web e Android.

O MeForma2, no que diz respeito ao software, veio com o objetivo de melhorar a portabilidade e a usabilidade da aplicação, permitindo que mais pessoas pudessem utilizá-la em seus dispositivos e que essa utilização fosse fácil e agradável. Para resolver as questões de portabilidade, foi desenvolvida uma versão Web do MeForma2, levando em consideração o fato de que aplicações Web podem ser executadas em qualquer dispositivo capaz de executar um navegador da Web. Para resolver as questões de usabilidade, um novo leiaute foi pensado com base em outras aplicações e nas opiniões dos usuários da primeira versão  do MeForma.

Dentre os desafios da versão Web estavam a responsividade da aplicação e a compatibilidade dos recursos em JavaScript, maior parte do aplicativo, com os navegadores disponíveis. Essas e outras questões foram resolvidas com o uso do Ionic Framework, um \textit{framework} para desenvolvimento de aplicativos híbridos responsivos. O Ionic é responsável por garantir a compatibilidade do código em JavaScript com as versões mais recentes dos navegadores mais populares do mundo. Além disso, o Ionic oferece recursos para que o aplicativo seja responsivo, ou seja seja compatível com qualquer tamanho de tela.

Pra mensurar a qualidade da aplicação, foram coletados dados utilizando as seguintes ferramentas: uma avaliação heurística, para avaliar a usabilidade da interface; uma avaliação de relevância da aplicação, para entender se a aplicação é útil em seu propósito; uma pesquisa de satisfação dos usuários, para avaliar o quão confortáveis os usuários estão com a utilização da aplicação; sugestões enviadas por usuários, para que a aplicação pudesse crescer com base na experiência de quem a utiliza; e estatísticas de uso da aplicação, para entender o comportamento dos usuários.

Como auxiliar na construção do MeForma2, foi construído um Web Scraping apelidado de CMF para explorar os dados de cursos, currículos e disciplinas da UFBA vistos como necessários para a aplicação. Graças ao CMF, o MeForma2 contemplou os 100 cursos de graduação da UFBA (sendo 93 cursos de Salvador e 7 de Vitória da Conquista), contra 79 da primeira versão; 112 currículos de cursos, contra 79 da primeira versão do MeForma; e 5419 disciplinas, contra 4966 da primeira versão.

O MeForma2 inclui um painel de monitoramento dos cursos apelidado de MFPAC, o qual utiliza os dados coletados pelo MeForma2 para oferecer à liderança dos cursos informações sobre o desempenho dos estudantes que possam embasar e auxiliar nas tomadas de decisão e execuções de ações administrativas a respeito dos cursos, além de contribuir com processos de orientação acadêmica individual.

O desenvolvimento do MeForma2 foi iniciado no dia 12 de Julho de 2018, e a aplicação foi liberada para o público no dia 28 de Outubro de 2018, Totalizando 109 dias de desenvolvimento, incluindo o dia inicial. Os dados coletados dos usuários do MeForma2 foram coletados entre os dias 28 de Outubro de 2018 e 16 de Novembro de 2018.
