% ---
% RESUMOS
% ---

% resumo em português
\setlength{\absparsep}{18pt} % ajusta o espaçamento dos parágrafos do resumo
\begin{resumo}
É difícil para o estudante ter ciência de sua evolução no curso de graduação sem o auxílio de uma ferramenta adequada. Pensando nisso, foi desenvolvido um sistema chamado MeForma. O MeForma tem o objetivo de apoiar e facilitar a vida acadêmica dos estudantes de graduação da Universidade Federal da Bahia, tornando-os mais participativos de sua própria formação, esclarecendo os passos necessários para atingí-la e ajudando-os a se orientar com relação a esses passos. Este trabalho traz uma reformulação completa do MeForma, com novas funcionalidades e correções de problemas como a limitação com relação ao alcance de usuários, problemas de usablidade e portabilidade, além de trazer uma avaliação sobre a qualidade da nova versão. A nova versão foi desenvolvida com preocupações modernas e recursos voltados a atender o maior número de estudantes de graduação possível.

 \textbf{Palavras-chave}: Universidade Federal da Bahia, formatura, extração de dados da web, aplicação híbrida, desempenho acadêmico.
\end{resumo}

% resumo em inglês
\begin{resumo}[Abstract]
 \begin{otherlanguage*}{english}
It is difficult for an student to be aware of his evolution in the undergraduate course without the aid of a suitable tool. Thinking about it, a system called MeForma was developed. MeForma aims to support and facilitate the academic life of undergraduate students at Federal University of Bahia, making them more participatory in their own training, clarifying the steps necessary to reach it and helping them to orient themselves in relation to those steps. This work brings a complete redesign of the MeForma, with new functionalities and fixes of problems such as limitation regarding user reach, usability and portability problems, as well as an evaluation of the quality of the new version. The new version was developed with modern concerns and resources aimed at serving as many undergraduate students as possible.

   \vspace{\onelineskip}
   \noindent 
   \textbf{Keywords}: Federal University of Bahia, graduation, web data extraction, hybrid apps, academic performance.
 \end{otherlanguage*}
\end{resumo}

%% resumo em francês 
%\begin{resumo}[Résumé]
% \begin{otherlanguage*}{french}
%    Il s'agit d'un résumé en français.
% 
%   \textbf{Mots-clés}: latex. abntex. publication de textes.
% \end{otherlanguage*}
%\end{resumo}
%
%% resumo em espanhol
%\begin{resumo}[Resumen]
% \begin{otherlanguage*}{spanish}
%   Este es el resumen en español.
%  
%   \textbf{Palabras clave}: latex. abntex. publicación de textos.
% \end{otherlanguage*}
%\end{resumo}
% ---