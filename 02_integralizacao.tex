\chapter{Justificativa}

O MeForma2 é um sistema cuja principal motivação é apoiar a vida acadêmica dos estudantes de graduação da UFBA, ajudando-os a se orientar com relação à conclusão de seus respectivos cursos, partindo da hipótese de que é difícil para o estudante ter ciência de sua evolução no curso de graduação sem o auxílio de uma ferramenta adequada. 

Nesse cenário, enquanto ferramenta de orientação, o MeForma2 serve para facilitar a passagem de um estudante pelo curso de graduação, deixando ele alerta sobre o que é necessário para concluir o currículo do curso e o quanto ele está progredindo semestre a semestre.

Com a inclusão do MFPAC, o MeForma2 se torna uma ferramenta muito útil para os colegiados e departamentos responsáveis por cada curso, pois através dos dados inseridos no MeForma2 pelos estudantes, é possível obter uma série de informações sobre o desempenho dos mesmos e visualizá-las no MFPAC. Essas informações podem contribuir para a execução de ações administrativas, por parte da liderança acadêmica, que melhorem os cursos, o desempenho dos estudantes e, por consequência, a avaliação dos cursos.

Os cursos de graduação no Brasil são avaliados pelo SINAES -- Sistema Nacional de Avaliação de Educação Superior. O SINAES Foi criado pela Lei n° 10.861, de 14 de abril de 2004, e é formado por três componentes principais: a avaliação das instituições, dos cursos e do desempenho dos estudantes. Os resultados da avaliação possibilitam traçar um panorama da qualidade dos cursos e instituições de educação superior no país.

Nesse cenário, o MeForma2 é uma ferramenta capaz de contribuir para que os cursos de graduação da UFBA melhorem sua avaliação perante os órgãos avaliadores a partir de insumos reais específicos sobre o desempenho dos estudantes em cada disciplina, podendo influenciar em adequações de grade curricular, oferta de disciplinas, escalonamento de professores, dentre outros aspectos aos quais os números gerados pelo sistema dizem respeito.