\chapter{Conclusão}

Este trabalho apresentou a aplicação MeForma2, seu funcionamento e sua importância para o acompanhamento do progresso dos estudantes de graduação da UFBA em seus respectivos currículos de curso, não só para os próprios estudantes, mas também para os gestores de cada curso.

O processo de desenvolvimento englobou a definição de requisitos, a escolha das linguagens de programação e das ferramentas utilizadas, a escolha das plataformas que receberiam suporte, a modelagem de um banco de dados, a implementação de crawlers, wrappers e da própria aplicação. O resultado final desse processo foi uma aplicação de uso simples e fácil, com uma execução dividida em etapas bastante definidas.

A aplicação foi avaliada por um grupo de usuários através de uma pesquisa de relevância, e estes a consideraram relevante não só para a organização de suas vidas acadêmicas, mas também para entender os pré-requisitos necessários para alcançar a formatura, além de contribuir para torná-los mais responsáveis com seu progresso no curso.

Uma avaliação heurística, mostrou que a aplicação não possui nenhum problema catastrófico de usabilidade e que está adequada para a utilização dos estudantes. Conclusão que foi reforçada pela pesquisa de satisfação realizada com um grupo de usuários que se mostraram contentes com o MeForma2, mesmo que para alguns a aplicação possua pontos que precisem de melhora.

As estatísticas de uso coletadas mostraram um aumento na quantidade de usuários da aplicação e a preferência dos usuários pela versão Web do MeForma2, mostrando que o desenvolvimento da versão web foi uma contribuição importante para os estudantes. Além disso, as estatísticas mostram que a versão web foi importante para o crescimento do MeForma2, pois conseguiu atingir grupos de usuários que o MeForma original não era capaz de atingir, como por exemplo, os usuários de dispositivos Apple.

Dentre as contribuições deste trabalho, se destacam:
\begin{itemize}
    \item O MeForma2 - que já está sendo utilizado pelos estudantes de graduação da UFBA sem restrições.
    \item O MFPAC - que está em versão alpha, sendo utilizado apenas pelo colegiado de Ciência da Computação da Universidade Federal da Bahia.
    \item A pesquisa de fontes de dados sobre disciplinas e cursos na UFBA - que podem ser utilizadas tanto para a melhoria do MeForma2, quanto para a criação de novas aplicações com objetivos diferentes.
    \item O CMF - O capítulo 4 mostrou os passos para construção do \textit{web scraping} CMF e como o mesmo trabalha para obter os dados de disciplinas e cursos da UFBA, o que pode guiar a criação de novos sistemas com fins similares.
\end{itemize}

Sobre o MFPAC e a leitura dos dados obtidos da utilização do MeForma2 pelos estudantes, é importante frizar que são ferramentas para auxiliar decisões administrativas, e que não traduzem a realidade com precisão, pois são apenas números. Números esses que podem levar a diferentes conclusões se forem analisados isoladamente, a depender do que está sendo avaliado. Uma decisão administrativa sobre uma disciplina, por exemplo, deve levar em consideração não apenas os números levantados, mas também a opinião dos envolvidos sobre a disciplina, a opinião dos estudantes sobre os professores que ensinam aquela disciplina, a qualidade do ambiente onde a disciplina é lecionada, o conteúdo da ementa, os tipos de avaliação que tem sido aplicados, e uma série de variáveis que influenciam no desempenho dos estudantes. 

Foram identificadas algumas possibilidades de trabalhos e funcionalidades que podem ser desenvolvidos futuramente para agregar mais valor tanto ao MeForma2 quanto aos seus usuários:

\begin{itemize}
    \item Permitir que o usuário realize a pré-matrícula semestral através do sistema - Facilitaria o trabalho dos gestores com relação ao planejamento das disciplinas que serão ofertadas em um determinado semestre.
    \item Sugestão de conjunto de disciplinas para o semestre seguinte - A intenção é sugerir ao estudante um semestre equilibrado no sentido de não se matricular em muitas disciplinas com alto índice de reprovação no mesmo semestre.
    \item Notificação sobre o lançamento da nota das disciplinas que o usuário está cursando no Siac - Agregaria valor à experiência do usuário.
    \item Agenda de aulas e atividades curriculares - A intenção é ajudar o estudante a se manter alerta sobre sua rotina e compromissos acadêmicos.
    \item Permitir simulação de cumprimento de disciplinas - Útil para auxiliar no planejamento dos semestres.
    \item Sugestão de formas de cumprir carga horária - Permitiria que estudantes que precisam cumprir carga horária descobrissem opções já utilizadas por outros.
    \item Corrigir os problemas apontados pela avaliação heurística.
\end{itemize}

O MeForma2 se mostrou adequado e relevante para os estudantes de graduação da UFBA e divulgá-lo é fundamental para que ele possa atingir seu objetivo com altos índices de qualidade, uma vez que, quanto mais usuários utilizarem o sistema, mais fiéis se tornam os dados que ele exibe. Além disso, o interesse de estudantes de outras instituições de ensino pela aplicação demonstram que a problemática que motivou seu desenvolvimento também está presente na vida acadêmica desses estudantes, e que o MeForma2 tem boas chances de ultrapassar as fronteiras da UFBA.